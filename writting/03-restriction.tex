\section{Access Control Restrictions}
CCN supports two types of content restrictions: {\tt KeyIdRestriction}
and {\tt ContentObjectHashRestriction}, or KeyID and ContentID, respectively.
The KeyID restricts content to that which is signed by a key whose hash matches
the KeyID. The ContentID restricts content to that which hashes to the ContentID.
Each of these restrictions can be used by any entity in the network, especially
routers, to verify content.

In this section, we advocate for a new restriction called an {\tt AccessRestriction},
or AccessID. The AccessID is designed to restrict content to that which satisfies
a given access control policy. An access control policy is one which binds a given
content to a specific decryption key. As such, only the owner of the decryption
can generate an AccessID. (This is necessary since, as we will show, AccessIDs serve
as committments to a requests. It must not be possible for a malicious user to forge
a committment to a request.) However, unlike the two prior restrictions, the AccessID
need not be verifiable by any entity in in the network. Confidentiality is an
end-to-end goal and not a feature provided by the network.

Given the obvious asymmetry in roles, this implies the use of a digital signatures
as the AccessID. For example, let $pk$ and $sk$ be public and private key pairs associated with
some access control policy and owned by consumer $Cr$. One trivial construction for AccessID
of $N$ with KeyID $k_ID$ is as follows:
$$
\mathsf{Sign}_{sk}(N || k_ID) || pk
$$
This would bind AccessID to $N$ and the producer's public key. Moreover, it would
allow any network entity to verify this restriction. However, in many cases, this
is overly revealing. Routers need not concern themselves with whether or not a
content matches the requested AccessID. A better construction would therefore limit
exposing this information to the network. Consider the following alternative:
$$
\mathsf{Sign}_{sk}(N || pk || k_ID) || H(N || pk || k_ID)
$$
This restricts verificaftion of the AccessID to those who know $pk$, which is not
otherwise revealed to the network.

An AccessID is inspired by the work of Ghali et al. \cite{ibac15} on
\emph{interest-based access control}. The idea is to bind principals to
interests and then make coupled authentication and authorization decisions
based on this binding. 
