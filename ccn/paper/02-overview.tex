\section{Preliminaries}\label{sec:prelims}
This section presents an overview of the CCN architecture\footnote{Named-Data Networking \cite{jacobson2009networking}
is an ICN architecture related to CCN. However, since CCNxKE was
designed for ICNs that have features which are not supported by NDN (such
as exact name matching), we do not focus on NDN in this work. However,
CCNx could be retrofitted to work for NDN as well.} and work
related to confidentiality, privacy, and transport security. Those familiar with these topics
can skip it without loss of continuity.

\subsection{CCN Overview}
In contrast to IP networks, which focus on end-host names and addresses,
CCN \cite{jacobson2009networking,mosko2016semantics} centers
on content by making it named, addressable, and routable within the network. A
content name is a URI-like string composed of one or more
variable-length name segments, each separated by a \url{`/'} character. To
obtain content, a user (consumer) issues a request, called an \emph{interest}
message, with the name of the desired content. This interest can be
\emph{satisfied} by either (1) a router cache or (2) the content producer. A
\emph{content object} message is returned to the consumer upon satisfaction of
the interest. Moreover, name matching in CCN is exact, e.g., an interest for
\url{/edu/uci/ics/cs/fileA} can only be satisfied by a content object
named \url{/edu/uci/ics/cs/fileA}.

In addition to a payload, content objects include several fields. In this work,
we are only interested in the following three: {\tt Name}, {\tt Validation}, and {\tt ExpiryTime}.
The {\tt Validation} field is a composite of (1) validation algorithm information
(e.g., the signature algorithm used, its parameters, and a link to the public
verification key), and (2) validation payload (e.g., the signature). We use the
term ``signature'' to refer to this field. {\tt ExpiryTime} is an optional,
producer-recommended duration for the content objects to be cached.
Conversely, interest messages carry a mandatory name, optional payload, and
other fields that restrict the content object response. The reader is encouraged
to review \cite{mosko2016semantics} for a complete description of all packet fields
and their semantics.

Packets are moved in the network by routers or forwarders. A forwarder is composed
of at least the following two components:
\begin{compactitem}
\item {\em Forwarding Interest Base} (FIB) -- a table of name prefixes and
  corresponding outgoing interfaces. The FIB is used to route interests based on
  longest-prefix-matching (LPM) of their names.
\item {\em Pending Interest Table} (PIT) -- a table of outstanding (pending)
  interests and a set of corresponding incoming interfaces.
\end{compactitem}
%
A forwarder may also maintain an optional {\em Content Store} (CS) used for content
caching. The timeout for cached content is specified in the \texttt{ExpiryTime}
field of the content header. From here on, we use the terms {\em CS} and
{\em cache} interchangeably.

Forwarders use the FIB to move interests from consumers towards producers and the PIT
to forward content object messages along the reverse path towards consumers. More
specifically, upon receiving an interest, a router $R$ first checks its cache
(if present) to see if it can satisfy this interest locally. If the content is
not in the cache, $R$ then consults the PIT to search for an outstanding
version of the same interest. If there is a PIT match, the new incoming interface
is added to the PIT entry. Otherwise, $R$ forwards the interest to the next hop
according to its FIB (if possible). For each forwarded interest, $R$ stores
some amount of state information in the PIT, including the name of the interest and the
interface from which it arrived, so that content may be sent back to the
consumer. When content is returned, $R$ forwards it to all interfaces listed in
the matching PIT entry and said entry is removed. If a router receives a content object
without a matching PIT entry, the message is deemed unsolicited and subsequently
discarded.
