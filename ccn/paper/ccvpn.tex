\documentclass[conference,letterpaper,10pt]{IEEEtran}
\usepackage{blindtext, graphicx, url, paralist}

% Add the compsoc option for Computer Society conferences.
%
% If IEEEtran.cls has not been installed into the LaTeX system files,
% manually specify the path to it like:
% \documentclass[conference]{../sty/IEEEtran}

% *** GRAPHICS RELATED PACKAGES ***
%
\ifCLASSINFOpdf
  % \usepackage[pdftex]{graphicx}
  % declare the path(s) where your graphic files are
  % \graphicspath{{../pdf/}{../jpeg/}}
  % and their extensions so you won't have to specify these with
  % every instance of \includegraphics
  % \DeclareGraphicsExtensions{.pdf,.jpeg,.png}
\else
  % or other class option (dvipsone, dvipdf, if not using dvips). graphicx
  % will default to the driver specified in the system graphics.cfg if no
  % driver is specified.
  % \usepackage[dvips]{graphicx}
  % declare the path(s) where your graphic files are
  % \graphicspath{{../eps/}}
  % and their extensions so you won't have to specify these with
  % every instance of \includegraphics
  % \DeclareGraphicsExtensions{.eps}
\fi
%
\usepackage{microtype}

\usepackage{subfigure}
\usepackage{fixltx2e}

\usepackage{multirow}

\usepackage{algorithm2e}
% \usepackage{algorithm}
\usepackage{algorithmic}
\usepackage{xcolor}
\newcommand\todo[1]{\textcolor{red}{#1}}

\hyphenation{op-tical net-works semi-conduc-tor}

\newcommand\adv{$\mathcal{A}$}

\newtheorem{definition}{\textbf{Definition}}[section]
\newtheorem{theorem}{\textbf{Claim}}[section]
\newtheorem{corollary}{\textbf{Corollary}}[theorem]
\newtheorem{lemma}[theorem]{\textbf{Lemma}}

\begin{document}

\title{Namespace Tunnels in Content-Centric Networks}

\author{
\IEEEauthorblockN{Ivan Oliviera Nunes, Gene Tsudik, Christopher A. Wood$^+$ \thanks{$^+$Supported by the NSF Graduate Research Fellowship DGE-1321846.}}
\IEEEauthorblockA{University of California Irvine \\
\{inunes, gene.tsudik, woodc1\}@uci.edu}
}

\maketitle


\begin{abstract}
Content Centric Networking (CCN) is a future Internet architecture which is envisioned as
an alternative to the current IP-based model. CCN emphasizes content distribution by making
contents directly addressable in a request-based information centric network. An advantage
of CCN is that it has some innate privacy friendly features, such as lack of source and
destination addresses in packets. However, to be considered a viable future Internet
architecture, CCN must offer services for private and anonymous communication that are
at least equivalent to those present in the IP architecture. Among such, VPNs are a very
popular application that enables users to send and receive data across shared or public
networks as if their computing devices were directly connected to the same private
network. In this work we design, implement and evaluate CCVPN, a content centric
virtual private network capable of providing the same functionality of VPNs within
the CCN Internet architecture. To the best of our knowledge CCVPN is the first proposal
of a VPN alike service for CCNs. In addition to the CCVPN design, we also provide a
security analysis and experimental performance evaluation for this new system.
\end{abstract}

\begin{IEEEkeywords}
Content-Centric Networking, VPN, Tunneling
\end{IEEEkeywords}

\IEEEpeerreviewmaketitle

\section{Introduction}

Content-centric networking (CCN) is a type of request-based information-centric
networking (ICN) architecture. In CCN, all data is named. Consumers obtain data
by issuing an explicit request for the content by its name. The network is
responsible for forwarding this request towards producers, based on the name.
The producers then generate and return the content response. Since a name uniquely identifies
a content response, routers may cache content responses and use them to satisfy
future requests for the same name.
% Since content may be served from anywhere
% in the network, all content has an (implicit or explicit) authenticator that
% is used to verify the name-to-data binding. In
% order to prevent content poisoning attacks, wherein a malicious producer supplies
% fake data in a content response that is propagated in the network, a router
% should never serve (a) content with an invalid authenticator or (b) cached
% content that it cannot verify.

One negative side effect of name-based requests is that any on-path or
eavesdropping adversary between a consumer and producer can learn the identity
and contents of all data in transit. In traditional IP-based networks, there are
generally two mechanisms to solve this problem: (1) anonymity
networks, such as Tor \cite{dingledine2004tor}, and (2) VPNs. As tools focused on anonymity,
the former help prevent linkability of packets to their requesters without
always protecting the identities or content themselves. In contrast, VPNs focus
on packet confidentiality by creating a tunnel between two private networks
or between a consumer and a single private network. All traffic over this tunnel is
encrypted and opaque to an eavesdropper. VPNs differ from anonymity
networks such as Tor in that they are \emph{network-layer} mechanisms that
typically only introduce a single layer of encryption to protect traffic.
Thus, while Tor can be used to enable VPN-like functionality, it is often
far more inefficient since it operates above the network layer. Moreover,
Tor is often too heavy for the simple use case of protecting packets from
prying eyes.

ANDaNA \cite{dibenedetto2011andana} was the first anonymity network to address these
network-layer privacy concerns in CCN.
Similar to Tor, ANDaNA uses circuits composed of anonymizing routers (ARs)
to marshall requests and responses between consumers and producers. The former
onion-encrypt interests and content using the public key(s) of the target ARs.
A variant of ANDaNA uses symmetric keys for packet encapsulation but suffers from
linkability. Tsudik et al. \cite{tsudik2016ac3n} proposed an optimized version of
the symmetric-key ANDaNA variant that did not permit linkability. Still, neither of
these systems were designed to address the simple use case of a VPN: protecting
data that should only be revealed within a private network. Though tunneling CCN
traffic may only be useful for a subset of applications, we believe it is a gap to be addressed
for this emerging technology, for a variety of reasons. {\bf First}, privacy
continues to be an elusive property for CCN applications. Tunneling will help
permit some degree of privacy within trusted AS domains from external passive
eavesdroppers. {\bf Second}, multi-hop circuits as used in ANDaNA are overkill
when the goal is packet privacy instead of anonymity. {\bf Third}, end-to-end
sessions, such as those enabled by CCNxKE \cite{ccnxke} and similar protocols,
only serve those engaged in the session. In contrast, since VPNs operate at the
network layer, tunneled traffic has the potential to serve any number of consumers
within the same trusted domain. Thus, while tunneling may contrast the content-centric
nature of data transmission in CCN, it fills a void for this architecture.

In this paper, we present CCVPN, a secure tunneling protocol and system design for
CCN. Similar to ANDaNA, CCVPN encrypts interests and content objects between two endpoints.
However, in CCVPN, the endpoints are network gateways instead of ARs. In the standard configuration, both
endpoints of the tunnel are gateways between two trusted domains. Tunnels may
also be nested. Moreover, it is possible for the source to be an individual consumer.
In fact, the standard two-hop ANDaNA circuit is identical to a nested tunnel
with the same source. Thus, CCVPN is more flexible than a general anonymity network.
CCVPN is designed to use efficient symmetric-key cryptographic primitives, but it also
works with public-key algorithms when a proper key exchange or distribution mechansim
is absent. This makes it easy to deploy CCVPN in real-world CCN networks.

We design, implement, and evaluate CCVPN to gauge its perceived impact on normal
traffic. Our results indicate that the collective throughput across multiple
consumers sharing a tunnel remains stable up to a modest bound of $60$ consumers, each
requesting data at a rate of $1$ mbps. Moreover, as expected, the average RTT of
consumer requests decreases at a rate proportional to the collective throughput
and request rate. Further improvements in both throughput and perceived RTT
can be made with an implementation in a more performant CCN router, such as that of the CICN
project \cite{cicn}.

The rest of this paper is organized as follows. Section \ref{sec:prelims} provides
an overview of CCN and the relevant protocol details necessary to understand
CCVPN. Section \ref{sec:related} describes previous related work that motivates
our design. Section \ref{metho} then describes the main CCVPN design with the necessary
cryptographic and packet format details. We analyze the security of CCVPN
in Section \ref{sec:sec-analysis}. We present the results of a comprehensive
analysis and experimental evaluation in Sections \ref{sec:analysis} and \ref{sec:exp},
respectively. We then conclude with a discussion of future work in Section \ref{sec:conclusion}

\section{Preliminaries}\label{sec:prelims}
This section presents an overview of the CCN architecture\footnote{Named-Data Networking \cite{jacobson2009networking}
is an ICN architecture related to CCN. However, since CCNxKE was
designed for ICNs that have features which are not supported by NDN (such
as exact name matching), we do not focus on NDN in this work. However,
CCNx could be retrofitted to work for NDN as well.} and work
related to confidentiality, privacy, and transport security. Those familiar with these topics
can skip it without loss of continuity.

\subsection{CCN Overview}
In contrast to IP networks, which focus on end-host names and addresses,
CCN \cite{jacobson2009networking,mosko2016semantics} centers
on content by making it named, addressable, and routable within the network. A
content name is a URI-like string composed of one or more
variable-length name segments, each separated by a \url{`/'} character. To
obtain content, a user (consumer) issues a request, called an \emph{interest}
message, with the name of the desired content. This interest can be
\emph{satisfied} by either (1) a router cache or (2) the content producer. A
\emph{content object} message is returned to the consumer upon satisfaction of
the interest. Moreover, name matching in CCN is exact, e.g., an interest for
\url{/edu/uci/ics/cs/fileA} can only be satisfied by a content object
named \url{/edu/uci/ics/cs/fileA}.

In addition to a payload, content objects include several fields. In this work,
we are only interested in the following three: {\tt Name}, {\tt Validation}, and {\tt ExpiryTime}.
The {\tt Validation} field is a composite of (1) validation algorithm information
(e.g., the signature algorithm used, its parameters, and a link to the public
verification key), and (2) validation payload (e.g., the signature). We use the
term ``signature'' to refer to this field. {\tt ExpiryTime} is an optional,
producer-recommended duration for the content objects to be cached.
Conversely, interest messages carry a mandatory name, optional payload, and
other fields that restrict the content object response. The reader is encouraged
to review \cite{mosko2016semantics} for a complete description of all packet fields
and their semantics.

Packets are moved in the network by routers or forwarders. A forwarder is composed
of at least the following two components:
\begin{compactitem}
\item {\em Forwarding Interest Base} (FIB) -- a table of name prefixes and
  corresponding outgoing interfaces. The FIB is used to route interests based on
  longest-prefix-matching (LPM) of their names.
\item {\em Pending Interest Table} (PIT) -- a table of outstanding (pending)
  interests and a set of corresponding incoming interfaces.
\end{compactitem}
%
A forwarder may also maintain an optional {\em Content Store} (CS) used for content
caching. The timeout for cached content is specified in the \texttt{ExpiryTime}
field of the content header. From here on, we use the terms {\em CS} and
{\em cache} interchangeably.

Forwarders use the FIB to move interests from consumers towards producers and the PIT
to forward content object messages along the reverse path towards consumers. More
specifically, upon receiving an interest, a router $R$ first checks its cache
(if present) to see if it can satisfy this interest locally. If the content is
not in the cache, $R$ then consults the PIT to search for an outstanding
version of the same interest. If there is a PIT match, the new incoming interface
is added to the PIT entry. Otherwise, $R$ forwards the interest to the next hop
according to its FIB (if possible). For each forwarded interest, $R$ stores
some amount of state information in the PIT, including the name of the interest and the
interface from which it arrived, so that content may be sent back to the
consumer. When content is returned, $R$ forwards it to all interfaces listed in
the matching PIT entry and said entry is removed. If a router receives a content object
without a matching PIT entry, the message is deemed unsolicited and subsequently
discarded.

\section{Related Work} \label{sec:related}

There are generally two classes of related work: (1) anonymity networks, such
as ANDaNA \cite{dibenedetto2011andana} and AC3N \cite{tsudik2016ac3n}, and
(2) encryption-based access control techniques. ANDaNA was developed as a
proof-of-concept application-layer anonymizing network for NDN. It works by
creating single-use, ephemeral, and anonymizing circuits between a consumer
and producer. Each hop in this circuit uses onion decryption to decapsulate
interests and onion encryption to encrypt the result. A minimum of two hops are
needed to guarantee consumer and producer unlinkability. CCVPN seeks to address
a different threat: privacy instead of anonymity. Thus, only a single hop, which
creates a tunnel between a source and a sink AS, is needed in CCVPN. Moreover,
since these tunnels serve multiple traffic flows for \emph{all} consumers within
the same source, they are long-lived and persistent. This reduces the per-packet
and per-flow cryptographic operations needed to move packets between the source
and sink domains. AC3N \cite{tsudik2016ac3n} is an optimized version of ANDaNA
that keeps per-flow state in each hop of an anonymizing circuit to specify
ephemeral key identifiers for packets. This prevents intra-flow linkability
while simultaneously enabling symmetric-key encryption and decryption for more
efficient processing. However, since AC3N is still a per-consumer application,
anonymizing circuits cannot be shared among multiple geolocated consumers
in the same domain.

Content encryption seeks to solve the problem of data confidentiality rather than
privacy or anonymity. This technique allows content to be disseminated throughout
the network, since it cannot be decrypted by adversaries without the appropriate decryption key(s).
Many variations of this approach have been proposed based on general
group-based encryption \cite{Smetters2010}, broadcast encryption \cite{Misra2013,Ion2013}, and
proxy re-encryption \cite{Wood2014}. Kurihara et al. \cite{ifip15} generalized these specialized
approaches in a framework called CCN-AC, an encryption-based access control framework
that shows how to use manifests to explicitly specify and enforce other encryption-based
access control policies. Consumers use information in the manifest to (1) request appropriate
decryption keys and (2) use them to decrypt content object(s). The NDN NBAC \cite{yu2015name}
scheme is similar to \cite{ifip15} in that it allows decryption keys to be
flexibly specified by a data owner. However, it does this based on name engineering rules instead of
configuration. Interest-based access control \cite{ghali2015interest} is a different
type of access control scheme wherein content is optionally encrypted. Access
is protected by making the names of contents derivable only by authorized consumers.
NDN-ACE \cite{shangndn} is a recent access control framework for IoT environments
which includes a key exchange protocol for distributing secret keys to sensors.
All of these techniques, with few exceptions, use public-key cryptographic schemes
to protect only the \emph{payload} of content packets. They do not encapsulate complete
packets for private transmission between a source and sink domain. The exception is CCNxKE~\cite{ccnxke},
which specifies a key exchange protocol that can bootstrap secure sessions between a source
a sink. CCNxKE can be used by CCVPN to establish pairwise shared secrets between tunnel
endpoints, even though this step is not strictly required.

\section{Hydra Design}

\subsection{Namespace Based Authorization}

There are at least two fundamental problems one must address in any access control
system: (1) how access control policies are represented and (2) how they are enforced.
Policy representation specifies how policies are mapped onto resources (or content).
For example, one representation might map content names to sets of public keys. These
keys could be owned by authorized consumers and are used when encrypting content.
Encrypting content with name $N$ under a key $pk$ restricts access to the owner of
the corresponding private key. The challenge is to devise a representation that
scales well with the number of resources (content) and consumers. Regardless of
the approach, we claim there is one fundamental feature that must be present for
every access control decision: consumer requests must be bound to a principal.
This allows the producer or entity serving data to provide the appropriate
representation of the target data to the consumer. To achieve this goal, Hydra
uses AccessIDs described in the previous section.

The second problem is rooted more in system design. Consider what must be done
to satisfy a request for access-controlled content. First, the requestor must
be authenticated to bind the request to a principal. Second, the request must
be authorized to determine if access to the desired content is permissable. Lastly,
the data must be packaged in a protected (encrypted) form for the requestor. Thus,
problem (2) is more about how entities are configured to handle the separate
authentication, authorization, content distribution steps in CCN.

In past work, these roles were often convoluted. In particular, \cite{pre,be}
assumed that the entity which handles data production was also implicitly
responsible for content authentication and authorization. CCN-AC \cite{xx} and
NDN-NBAC \cite{xx} separated the authentication and authorization service from
the data production. Specifically, data owners generate and distribute consumer (principal)
private keys to consumers through out-of-band channels. Data producers receive
the corresponding public keys through a similar channel. These are used to encrypt
randomly generated per-content encryption keys.

Despite this separation, these designs still suffer from the following problems.
{\bf First}, authentication and authorization are unnecessarily coupled, leading
to an inherent ``time of check'' versus ``time of use'' problem. Specifically,
consumers obtain private keys from the data owner at time $t$ and use them to
decrypt content at time $t' > t$. {\bf Second}, if consumers are forced to fetch
their private keys from the data owner, a single point of failure emerges. In
particular, it becomes easy to launch a Denial of Service (DoS) attack on the
data owner by forcing it to perform expensive cryptographic operations, e.g.,
signature verifications.
%% @Ivan, the reasons above are ridiculous -- please add others!

\subsection{Protocol}

Hydra is a system designed to address these problems. It has the following features:
%
\begin{compactitem}
    \item Consumer authentication and content authorization decisions can be separated
    and performed by separate systems in the network. A Hydra administrator can spawn
    any number of authentication endpoints to handle client authentication requests.
    Consumers are unable to fetch data from the authorization agent without having
    first been authenticated.
    \item Authorization decisions are centralized to a single system (or set of synchronized
    systems). This permits each authorization check to be done in real-time without
    introducing any added delay between the subsequent use.
    \item XXX
    \item XXX
    \item XXX
\end{compactitem}
%

\begin{table}
\centering
\caption{Notation summary}
\label{notation}
\begin{tabular}{|l|p{6cm}|}
\hline
Notation    			&  Description  							\\ \hline \hline
$I.name$			&  Name of the issued interest I					\\&\\
$N$				&  A namespace prefix (e.g., /uci/ics/ivan/\*) 				\\&\\
$TGT\_Name$			&  Ticket-granting-ticket name (e.g., /uci/ics/TGT) that will be routed towards authenticator \\&\\
$TGS\_Name$			&  Ticket-granting-service name (e.g., /uci/ics/TGS) that will be routed towards authorizer   \\&\\
$sk_C$      			&  Consumer Secret Key						        \\&\\
$pk_C$			        &  Consumer Public Key, including public UID and certificate        	\\&\\
$k_A$ 	   		 	&  Long term symmetric key shared between Authenticator and Authorizer  \\&\\
$k_P$ 	   		 	&  Long term symmetric key shared between Authenticator and Producer    \\&\\
$s \sample \{0,1\}^{\lambda}$	&  Random ${\lambda}$-bits number generation    	     		\\&\\
$ct = Enc_{k}(pt) $		&  Authenticated Encryption of $pt$ using symmetric key $k$		\\&\\
$pt = Dec_{k}(ct) $		&  Decryption of $ct$ using symmetric key $k$    	     		\\&\\
$ct = Enc_{pk}(pt) $		&  Authenticated Encryption of $pt$ using public key $pk$ 		\\&\\
$pt = Dec_{sk}(ct) $		&  Decryption of $ct$ using secret key $sk$    	     			\\&\\
$\sigma = Sign_{sk}(m) $	&  Signature on message $m$ using secret key $sk$ 			\\&\\
$Verif_{pk}(\sigma,m) $		&  Signature verification using public key $pk$     			\\&\\
\hline
\end{tabular}
\end{table}

\begin{figure}
\begin{center}
\includegraphics[width=\columnwidth]{Figures/hydra.pdf}
\caption{Hydra system design}
\label{fig:spectr-basic}
\end{center}
\end{figure}




\begin{figure*}
\begin{center}
\fbox{
\procedure{}{%
\textbf{Consumer} \> \> \textbf{Authenticator} \\
\mathsf{I.name = TGT\_Name} \> \> \\
\sigma \gets \mathsf{Sign}_{sk_C}(\mathsf{userName}) \> \> \\
\> \xrightarrow{payload = \sigma, userName} \> \\
\> \> \text{Fetch $pk_C$ according to userName} \\
\> \> \mathsf{Verify}_{pk_C}(\sigma,\mathsf{userName}) \\
%\> \> s \sample \{0,1\}^{\lambda} \\
\> \> t_1 \gets \mathsf{setTGTExpiration}() \\
\> \> k_{TGS} \sample \{0,1\}^{\lambda} \\
\> \> \mathsf{token}_{TGS}^{C} \gets \mathsf{Enc}_{pk_C}(k_{TGS}||t_1) \\
\> \> \mathsf{TGT} \gets \mathsf{Enc}_{k_{A}}(\mathsf{userName} || t_1 || k_{TGS}) \\
\> \xleftarrow{payload = \mathsf{TGT}, \mathsf{token}_{TGS}^{C}} \> \\
k_{TGS}||t_1 \gets \mathsf{Dec}_{sk_C}(\mathsf{token}_{TGS}^{C}) \> \> \\
\mathsf{store(TGT,t_1,k_{TGS})} \> \> \\
}
}
\caption{Consumer authentication protocol}
\label{fig:spectr-basic}
\end{center}
\end{figure*}




\begin{figure*}
\begin{center}
\fbox{
\procedure{}{%
\textbf{Consumer} \> \> \textbf{Authorizer} \\
\mathsf{I.name = TGS\_Name} \> \> \\
\> \xrightarrow{payload=N, \mathsf{TGT}} \> \\
\> \> \mathsf{userName} || t_1 || k_{TGS} \gets \mathsf{Dec}_{k_{A}}(TGT) \\
%\> \> \text{Verify: $s = s'$} \\
\> \> \text{Verify: $t_1$ not expired} \\
\> \> {k_P} \gets \mathsf{verifyPolicyAndFetchKey}(N, \mathsf{userName}) \\ %Kp can be different things (shared key between all boxes that implement that producer, broadcast encryption key, etc)
\> \> k_{N} \sample \{0,1\}^{\lambda} \\
%\> \> r \sample \{0,1\}^{\lambda} \\ why?
\> \> t_2 \gets \mathsf{setTGSExpiration}() \\ %% timestamp
\> \> \mathsf{TGS} \gets \mathsf{Enc}_{k_P}( N || k_N || t_2) \\ %% encrypt the key in the ticket
\> \> \mathsf{token}_N^{C} \gets \mathsf{Enc}_{k_{TGS}}(k_N||t_2) \\
\> \xleftarrow{payload= \mathsf{TGS}, \mathsf{token}_N^{C}} \> \\
k_N || t_2 \gets \mathsf{Dec}_{k_{TGS}}(\mathsf{token}_N^{C}) \> \\
\mathsf{store(N,TGS,t_2,k_N)} \>
}
}
\caption{Consumer-data authorization protocol}
\label{fig:spectr-basic}
\end{center}
\end{figure*}

\begin{figure}
\begin{center}
\fbox{
\procedure{}{%
\textbf{Consumer} \> \> \textbf{Producer} \\
\mathsf{I.name = N||suffix} \> \> \\
\> \xrightarrow{payload= TGS} \> \\
\> \>  N' || k_N || t_2 \gets \mathsf{Dec}_{k_P}(TGS) \\
\> \> \text{Verify: $N'$ is prefix of I.name} \\
%%\> \> \text{Verify: $pk_C'$ = $pk_C$} \\ Not sure this is necessary
\> \> \text{Verify $t_2$ expiration} \\
\> \> D \gets \mathsf{ProduceData}(\mathsf{I.name}) \\
\> \> D' \gets \mathsf{Enc}_{k_N}(D) \\
\> \xleftarrow{payload=D'} \> \\
D \gets \mathsf{Dec}_{K_N}(\mathsf{D'}) \> \> \\
}
}
\caption{Authorization verification protocol}
\label{fig:spectr-basic}
\end{center}
\end{figure}

\begin{figure}
\begin{center}
\fbox{
\procedure{}{%
\textbf{Consumer} \> \> \textbf{Producer} \\
\mathsf{I.name = N||suffix} \> \> \\
\> \xrightarrow{payload= TGS} \> \\
\> \>  N' || k_N || t_2 \gets \mathsf{Dec}_{k_P}(TGS) \\
\> \> \text{Verify: $N'$ is prefix of I.name} \\
%%\> \> \text{Verify: $pk_C'$ = $pk_C$} \\ Not sure this is necessary
\> \> \text{Verify $t_2$ expiration} \\
\> \> D \gets \mathsf{ProduceData}(\mathsf{I.name}) \\
\> \> D' \gets \mathsf{Enc}_{k_N}(D) \\
\> \xleftarrow{payload=D'} \> \\
D \gets \mathsf{Dec}_{K_N}(\mathsf{D'}) \> \> \\
}
}
\caption{Authorization verification protocol including challenge-based consumer authentication}
\label{fig:spectr-basic}
\end{center}
\end{figure}


\subsection{Ivan: Comments}
\begin{enumerate}
 \item IMO, the authentication algorithm (Fig.2) should be unrelated to the namespace you want to get access to. Only related to Consumer's claimed Identity. Otherwise the consumer has to go back to the Authenticator every time she wants a different content (3 RTT).
 \item Unless we are assuming that there is some magical non-deterministic router that will route the same name through different paths, I.name must specify the content produced by the Authenticator, i.e., the TGT. The same applies for TGS.
 \item The TGT as a MAC never expires. My suggestion is to encrypt a timestamp, as in Kerberos, and as we are doing in the authorizer. MAC with an epoch is also an option, but not a good one.
 \item PKE of $K_N$ in the authorization algorithm (Fig 3) could (and maybe should) be symmetric AEAD.
 \item $verifyPolicyAndFetchKey()$ : Could be implemented as broadcast encryption...
 \item Consumer has to receive the expiration date of TGT/TGS. This way Consumer can get back to Authenticator/Authorizer directly, without issuing expired TGT/TGS messages to authorizer/producers.
\end{enumerate}
\input{05-sec-analysis}
\input{06-perf-analysis}
\input{07-experiment}

\section{Conclusion}\label{sec:conclusion}
We presented the design, implementation, and assessment of CCVPN, a scheme and
protocol for building secure and private tunnels in CCN. CCVPN allows two namespaces
to be bridged by a tunnel across a public network. Unlike point-to-point tunnels,
such as those enabled by secure session protocols, CCVPN allows many consumers
to share the tunnel to access the private namespace. CCVPN is designed with efficiency
in mind: public-key cryptographic operations are kept to a minimum during normal
operation; symmetric-key packet encapsulation algorithms are used to marshall
packets between gateways. Our experimental evidence indicate that CCVPN offers
modest performance in the presence of a variable number of consumers and producers.
The results suggest that CCVPN could be deployed for private namespace tunneling
in real world CCN deployments.

For future work, we plan to implement the CCNxKE protocol from~\cite{ccnxke} as a
way to bootstrap symmetric-key tunnels in CCVPN. We will also integrate CCVPN into
real world testbeds to assess its performance for real-world applications. These
include video streaming and large-scale content dissemination applications. Finally,
we plan to investigate countermeasures to DoS attacks on CCVPN, such as puzzles for tunnel establishment.

\ifCLASSOPTIONcaptionsoff
  \newpage
\fi

\tiny

\bibliographystyle{IEEEtran}
\bibliography{references}

% \begin{IEEEbiography}[{\includegraphics[width=1in,height=1.25in,clip,keepaspectratio]{picture}}]{John Doe}
% \blindtext
% \end{IEEEbiography}

\end{document}
