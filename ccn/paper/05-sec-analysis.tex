\section{CCVPN Security}\label{sec:sec-analysis}


\subsection{System and Security Model} \label{sec:goals}

Our security analysis considers the worst case scenario, in which the consumer issues an
interest that has not been cached in any of the routers from the consumer nor producer domains.
Therefore, the interest must traverse the entire network until reaching the producer.
Conversely, the content needs to be forwarded all the way back though the same path
satisfying the routers' PIT entries. A system that is secure in this worst case is also
secure when the content has been previously cached in some router along the way.

\textbf{Adversary Goals and Capabilities.} The goal of the adversary \adv\ is either 1) to
retrieve any information associated with the original issued interest $I_p$ and/or the
original content $C_p$ or 2) to retrieve the identities of the consumer and/or producer,
thus violating the anonymity in the communication. \adv\ is also considered successful
if it is able to impersonate the content producer by faking a content response. Since the
original interest and the original content are visible inside the consumer and producer
domains, we here assume that \adv\ cannot eavesdrop or compromise hosts inside
such domains. It is also worth to emphasize that, by the conceptual design of the VPN,
communication is anonymous to hosts from outside the VPN but not to hosts that belong to
that VPN. Anyhow, one-to-one anonymous communication is still achievable sing the VPN
paradigm if the communication must be anonymous to any other host in the network. This
also means that, as components of the producer and consumer domains, we assume that the
gateways $G_c$ and $G_p$ cannot be compromised.

In our model \adv\ is allowed to perform the following actions (when outside the consumer and producer domains):

\begin{itemize}
	\item \textbf{Eavesdrop and replay traffic:} \adv\ can eavesdrop on a link
    learning among other things the packet contents and traffic patterns.
	\item \textbf{Deploy compromised routers or compromise existing routers:} \adv\ is capable of
    deploying a compromised router ou compromised an existent router outside the VPN domains. By doing
    so \adv\ becomes capable of maliciously injecting, delaying, altering, and dropping traffic.
    In addition, when an existing router is compromised, \adv\ learns all of the private
    information contained in that router, such as private keys and cached content.
	\item \textbf{Deploy compromised caches:} As a consequence of the ability to compromise and
    deploy compromised routers, \adv\ is also able to deploy compromised caches. This
    includes monitoring the routers' cache contents and replying with corrupted or fake data.
\end{itemize}

\subsection{Additional Considerations}

\textbf{Gateway-to-Gateway Authentication:} At this point one might have noticed that,
in the CCVPN design, any host that possesses the producer -side gateway ($G_p$) public-key
is able to initiate an anonymous communication link with $G_p$. In other words, the design
does not include any authentication between the consumer-side gateway $G_c$ and $G_p$.
Indeed, authentication is not required in CCVPN because is in not needed in some application scenarios.

Suppose, for instance, that a content provider offers its contents to any host in the Internet
but it also  wants such hosts to be able to anonymously request and receive such content. In
this case, since any host in the Internet should be able to requests the contents, there is no
need for the consumer-side $G_c$ to authenticate it self to $G_p$.

Another use case for CCVPN is the traditional VPN use case, in which two physically separated
local networks (for instance, offices of the same company in different countries) should virtually
behave as a private local network. In that case, it is necessary to prevent that a given $G_c$,
which is not part of that corporation's network to connect to $G_p$ and become part of the VPN.
To that purpose, standard host-to-host CCN authentication mechanisms can be used to authenticate
$G_p$ and $G_c$ to each other. This must be performed before any VPN communication, to make sure
that only the appropriate parties are communicating under the CCVPN architecture. We leave the
evaluation and specification of gateway-to-gateway authentication protocols as future work (see Sec.~\ref{sec:conclusion}).

\textbf{Denial of Service:} Since CCVPN gateways are connected to the public network they are
clearly susceptible to DoS attacks. In the CCVPN architecture, both $G_c$ and $G_p$ are
susceptible to DoS attacks. A DoS attack on $G_p$ would consist in flooding it with several
fake encapsulated interests. Conversely, a DoS attack on $G_c$ would basically consist
of flooding it with an enormous amount of encrypted content packets. In the case of $G_p$,
such attacks are specially harmful since the interest decapsulation involves a public-key
decryption operation. We plan to consider DoS countermeasures in future work.

\subsection{Security Analysis}

In this section we analyze the security of CCVPN. Our main security goal is
preventing \adv\ from achieving any of the goals outlined in Section \ref{sec:goals}.
Formally, this translates into semantic security of all traffic, modulo what can be inferred via traffic analysis.
A consequence of this property is that an off-path adversary, i.e., one that is
not in the consumer or producer domain, is unable to forge packets with
non-negligible probability. Our analysis relies on arguments in the standard
security model. It consists of assessing the security of the interest and
content encapsulation algorithms.

\begin{definition}\label{def1}
\textit{
An interest encapsulation algorithm $Encapsulate(I_p)$ is an indistinguishable
interest encapsulation iff, given any two interests $I_p^1$ and $I_p^2$, chosen
by the adversary, and a randomly selected bit $b$, the adversary has only $1/2 + \epsilon$
probability of guessing the value of the bit $b$ when given $I_e^b = Encapsulate(I_p^b)$.
Where $\epsilon$ is a negligible factor with regard to the security parameter $k$.
}
\end{definition}

\begin{theorem}\label{theo1}
\textit{
Let $Encapsulate_{pk}(I_p)$ denote the interest encapsulation routine described in
Algorithm~\ref{alg:interestEncap}. If $Enc_{pk}$, used to construct $Encapsulate_{pk}(I_p)$,
is a CPA-secure public-key encryption scheme then $Encapsulate_{pk}(I_p)$ is an
indistinguishable interest encapsulation algorithm.
}
\end{theorem}

\textit{\textbf{Proof--}} Suppose that Claim~\ref{theo1} is false. Then there exists a
polynomial adversary $Adv$ capable of guessing the bit $b$ of Definition~\ref{def1}
with non-negligible advantage, when given $I_e^b = Encapsulate(I_p^b)$ with
$b \leftarrow \{0,1\}$ chosen at random. We show that if such adversary exists he can
be used to construct a polinomial adversary $AdvCPA$ which breaks the CPA-security
of $Enc_{pk}$. $AdvCPA$ plays the CPA-security game with a challenger sending him
two messages $m^0$ and $m^1$. Following the CPA-security game, the challenger
randomly chooses a value for the bit $b' \leftarrow \{0,1\}$ and gives back
$C = Enc_{pk}(m^{b'})$ to $AdvCPA$. To break the CPA-security $AdvCPA$ must be
able to guess the value of the bit $b'$ with non-negligible advantage. To that
purpose $AdvCPA$ can query the challenger for the encryptions of $m^0$ and
$m^1$ ($c^0 = Enc_{pk}(m^0)$ and $c^1 = Enc_{pk}(m^1)$) and then construct two
interests $I_e^0 = createNewInterest(Gp_{name}, c^0)$ and $I_e^1 = createNewInterest(Gp_{name}, c^1)$,
using the same $createNewInterest$ function used by algorithm~\ref{alg:interestEncap},
which is public (notice that $Gp_{name}$ is also public). Finally, $AdvCPA$ gives $I_e^0$
and $I_e^1$ as input to $Adv$ and outputs whatever $Adv$ outputs. Since under our
assumption $Adv$ guesses the bit $b$ with non-negligible advantage, then $AdvCPA$ breaks
the CPA-security of $Enc_{pk}$. But this violates the hypothesis of Claim~\ref{theo1}
and, therefore, such $Adv$ cannot exist.

\begin{definition}
\textit{
A content encapsulation algorithm $Encapsulate(C_p)$ is an indistinguishable content
encapsulation iff, given any two contents $C_p^1$ and $C_p^2$, chosen by the adversary,
and a randomly selected bit $b$, the adversary has only $1/2 + \epsilon$ probability
of guessing the value of the bit $b$ when given $C_e^b = Encapsulate(C_p^b)$. Where
$\epsilon$ is a negligible factor with regard to the security parameter $k$.
}
\end{definition}

\begin{theorem}
\textit{
Let $ContentEnc_{k_r}(C_p)$ denote the content encapsulation routine described in
Algorithm~\ref{alg:contentEnc}. If $EncryptThenMAC_{k_r}$, used to construct
$ContentEnc_{sk}$, is a CCA-secure symmetric-key encryption scheme, then:
\begin{enumerate}
\item $ContentEnc_{k_r}(C_p)$ is an indistinguishable content encapsulation algorithm;
\item An adversary has only negligible probability of generating a valid fake encapsulated content $I_c'$
\end{enumerate}
}
\end{theorem}

\textit{\textbf{Proof (Sketch)--}} This follows directly from the definition of
CCA-security and from the same argument in the previous proof.
