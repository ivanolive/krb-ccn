\section{Introduction}

Content-centric networking (CCN) is a type of request-based information-centric
networking (ICN) architecture. In CCN, all data is named. Consumers obtain data
by issuing an explicit request for the content by its name. The network is
responsible for forwarding this request towards producers, based on the name.
The producers then generate and return the content response. Since a name uniquely identifies
a content response, routers may cache content responses and use them to satisfy
future requests for the same name.
% Since content may be served from anywhere
% in the network, all content has an (implicit or explicit) authenticator that
% is used to verify the name-to-data binding. In
% order to prevent content poisoning attacks, wherein a malicious producer supplies
% fake data in a content response that is propagated in the network, a router
% should never serve (a) content with an invalid authenticator or (b) cached
% content that it cannot verify.

One negative side effect of name-based requests is that any on-path or
eavesdropping adversary between a consumer and producer can learn the identity
and contents of all data in transit. In traditional IP-based networks, there are
generally two mechanisms to solve this problem: (1) anonymity
networks, such as Tor \cite{dingledine2004tor}, and (2) VPNs. As tools focused on anonymity,
the former help prevent linkability of packets to their requesters without
always protecting the identities or content themselves. In contrast, VPNs focus
on packet confidentiality by creating a tunnel between two private networks
or between a consumer and a single private network. All traffic over this tunnel is
encrypted and opaque to an eavesdropper. VPNs differ from anonymity
networks such as Tor in that they are \emph{network-layer} mechanisms that
typically only introduce a single layer of encryption to protect traffic.
Thus, while Tor can be used to enable VPN-like functionality, it is often
far more inefficient since it operates above the network layer. Moreover,
Tor is often too heavy for the simple use case of protecting packets from
prying eyes.

ANDaNA \cite{dibenedetto2011andana} was the first anonymity network to address these
network-layer privacy concerns in CCN.
Similar to Tor, ANDaNA uses circuits composed of anonymizing routers (ARs)
to marshall requests and responses between consumers and producers. The former
onion-encrypt interests and content using the public key(s) of the target ARs.
A variant of ANDaNA uses symmetric keys for packet encapsulation but suffers from
linkability. Tsudik et al. \cite{tsudik2016ac3n} proposed an optimized version of
the symmetric-key ANDaNA variant that did not permit linkability. Still, neither of
these systems were designed to address the simple use case of a VPN: protecting
data that should only be revealed within a private network. Though tunneling CCN
traffic may only be useful for a subset of applications, we believe it is a gap to be addressed
for this emerging technology, for a variety of reasons. {\bf First}, privacy
continues to be an elusive property for CCN applications. Tunneling will help
permit some degree of privacy within trusted AS domains from external passive
eavesdroppers. {\bf Second}, multi-hop circuits as used in ANDaNA are overkill
when the goal is packet privacy instead of anonymity. {\bf Third}, end-to-end
sessions, such as those enabled by CCNxKE \cite{ccnxke} and similar protocols,
only serve those engaged in the session. In contrast, since VPNs operate at the
network layer, tunneled traffic has the potential to serve any number of consumers
within the same trusted domain. Thus, while tunneling may contrast the content-centric
nature of data transmission in CCN, it fills a void for this architecture.

In this paper, we present CCVPN, a secure tunneling protocol and system design for
CCN. Similar to ANDaNA, CCVPN encrypts interests and content objects between two endpoints.
However, in CCVPN, the endpoints are network gateways instead of ARs. In the standard configuration, both
endpoints of the tunnel are gateways between two trusted domains. Tunnels may
also be nested. Moreover, it is possible for the source to be an individual consumer.
In fact, the standard two-hop ANDaNA circuit is identical to a nested tunnel
with the same source. Thus, CCVPN is more flexible than a general anonymity network.
CCVPN is designed to use efficient symmetric-key cryptographic primitives, but it also
works with public-key algorithms when a proper key exchange or distribution mechansim
is absent. This makes it easy to deploy CCVPN in real-world CCN networks.

We design, implement, and evaluate CCVPN to gauge its perceived impact on normal
traffic. Our results indicate that the collective throughput across multiple
consumers sharing a tunnel remains stable up to a modest bound of $60$ consumers, each
requesting data at a rate of $1$ mbps. Moreover, as expected, the average RTT of
consumer requests decreases at a rate proportional to the collective throughput
and request rate. Further improvements in both throughput and perceived RTT
can be made with an implementation in a more performant CCN router, such as that of the CICN
project \cite{cicn}.

The rest of this paper is organized as follows. Section \ref{sec:prelims} provides
an overview of CCN and the relevant protocol details necessary to understand
CCVPN. Section \ref{sec:related} describes previous related work that motivates
our design. Section \ref{metho} then describes the main CCVPN design with the necessary
cryptographic and packet format details. We analyze the security of CCVPN
in Section \ref{sec:sec-analysis}. We present the results of a comprehensive
analysis and experimental evaluation in Sections \ref{sec:analysis} and \ref{sec:exp},
respectively. We then conclude with a discussion of future work in Section \ref{sec:conclusion}
