\section{Performance Analysis}\label{sec:analysis}

In this section we analyze and discuss the overhead of the CCVPN design with respect
to the additional processing time and state consumption needed to handle traffic.

\subsection{State Consumption}
The CCVPN design has an immediate impact on the FIB and PIT size of a gateway.
(The content store size remains unaffected since only decapsulated content objects
are ever cached.) Let $F_S$ be the total size of a standard forwarder
FIB in terms of bytes and
$N_F$ be the number of entries in the FIB. For simplicity, we will assume that
each name prefix in the FIB has a constant size of $64$B. In practice we expect
this to be a comfortable upper bound. Thus, $F_S = N_Fs$, where $s$ is the size of
each FIB entry. Here, $s$ includes a name prefix (of size $64$B) and a bit vector
that identifies the matching links for the interface. We assume that a gateway has
$128$ links which, again, is a comfortable upper bound. Therefore, $F_S = 80N_F$B.
Now consider the FIB size $F_G$ for a CCVPN gateway. Some entries in these FIBs will
point to ``private'' prefixes, i.e., other domains, and therefore have a larger size
to account for the corresponding prefix and key material that must be stored.
For both public- and symmetric-key encryption, the key size is the same: $32$B \cite{sodiumGithub}.
Therefore, by taking into account two both the FIB entry prefix key, translation
prefix, encryption key, and corresponding bit vector, the total size of one ``private''
FIB entry will be $176$B, meaning that $F_G = 176N_F$B. By comparing $F_S$ to $F_G$, we
see that, in the worst case, the CCVPN FIB is at most $F_G/F_S = 176/80 = 2.2$ times larger
than the standard FIB. In practice, however, we expect this to be much smaller, since the
fraction of public to private FIB entries in a gateway will be non-zero.

We will now apply the same analysis to the PIT size. A standard PIT entry includes a
complete name and ingress bit vector. (They may also include the optional {\tt KeyId}
and {\tt ContentId}, but since they are included in the gateway PIT as well we omit them
from this analysis.) A gateway PIT entry will contain the same elements of a standard
PIT entry but also a symmetric encryption key ($32$B), nonce ($12$B), and an
encapsulation name ($64$B + $32$B).
The encapsulation name is the name of an encapsulated interest and includes an additional
$32$B {\tt PayloadID} segment to identify the encapsulated value in the payload. Let
$P_S$ and $P_G$ be the sizes of the standard and gateway PIT, respectively, and let $N_P$
be the number of PIT entries in one such table. Based on the above discussion, and assuming
again that a name is at most $64$B, a standard PIT entry is of the size $80$B. In contrast,
a gateway PIT entry is of size $204$B. Therefore, in the worst case, the CCVPN PIT
will be at most $P_G / P_S = 204/80 = 2.55$B larger than the standard PIT. Assuming
a steady state size of approximately $1e^5$ entries \cite{carofiglio2015pending},
this means that the PIT will be $20.4$MB, which is well within the capacity of
modern systems.

\subsection{Processing Overhead}
In terms of processing overhead, the gateway adds a number of new steps to the data
path of a packet. The main computational burdens are packet encapsulation and decapsulation.
In the public-key variant of CCVPN, interests are processed using public-key encryption,
whereas content is always processed using symmetric-key encryption. Let $T_E^P(n)$ and $T_D^P(n)$
be the time to encrypt and decrypt $n$B of data using a suitable public-key encryption scheme.
Similarly, let $T_E^S(n)$ and $T_D^S(n)$ be the time to encrypt and decrypt $n$B of data
using a symmetric-key encryption scheme. Then, the latency in a single interest-content
exchange is increased by $T = T_E^P(n_I) + T_D^P(n_I) + T_E^S(n_C) + T_D^S(n_C)$, where
$n_I$ and $n_C$ are the original interest and content sizes, respectively. As a rough
estimate, \cite{benchmarks} lists the cost of AES-GCM to be $2.946\mu s$ for setup
followed by $102$MiB/second Intel Core 2 1.83 GHz processor under Windows Vista in
32-bit mode (with AES ISA support). For packets that are at most $1500$B,
the total processing time is roughly $17\mu s$. Moreover, The public-key encryption and
decryption operations will always be at least as expensive, so the total latency is
increased by at least $T = 4 \times 17\mu s = 68 \mu s$. In comparison to the network
latency for a single packet this may not be noticable, but for a steady arrival state of
approximatey $1e^5$, this would lead to an instable system that would quickly overflow.
(This is because $65 \mu s \times 1e^5 = 6.8s$.) Therefore, there is an upper bound on
the number of private packets a gateway can process per second. This bound is entirely
dependent on the system configuration and network conditions.

Another performance deficiency comes from the fact that gateways cannot process packets
without allocating memory. Specifically, each packet requires either an encryption
or decryption. However, this cannot be done entirely in-place, the gateway must allocate some amount
of memory for every processed packet, e.g., to store the MAC tag, to account for
ciphertext expansion, or to apply padding, if desired. This overhead can outweigh
the cryptographic computations if the packet arrival rate is high enough. Therefore,
when implementing CCVPN, special care must be taken to ensure that memory allocations
are minimized or avoided where possible.
