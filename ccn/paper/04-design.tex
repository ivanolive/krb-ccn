\section{CCVPN Design}\label{metho}

Traditional Virtual Private Networks (VPNs) extend private networks across the Internet.
They enable users to send and receive data across shared or public networks as if their
computing devices were directly connected to the same private network~\cite{khanvilkar2004virtual}.
The goal of CCVPN it to provide its users with the same functionality within a CCN Internet architecture.
% Therefore, the users can benefit from the features and security of a private network, even though
% they are not physically under the same private network.

In CCVPN, there are four main entities involved in the virtual private communication: \textit{Consumer}, \textit{Producer}, \textit{Consumer Side Gateway} ($G_c$), and \textit{Producer Side Gateway} ($G_p$).
As usual in CCNs, the \textit{Consumer} is the network node that issues an interest for a given content
(e.g., file, web page, video) that it wishes to retrieve. The Producer is the network node which is
capable of providing such content. $G_c$ and $G_p$ are the edge gateways responsible for ensuring the
private communication among distinct domains. As we discuss later on, both $G_c$ and $G_p$ process can
run in a single gateway to enable bidirectional content service between two domains. However, we
firstly present them as separate entities for the purpose of clarity.

\begin{figure}
\centering
\includegraphics[width=\columnwidth]{images/architecture.pdf}
\caption{CCVPN connectivity architecture}
\label{fig:ccvpn}
\end{figure}

As depicted in Fig.~\ref{fig:ccvpn}, the devices inside the \textit{Consumer} domain form a physically
interconnected private network. Conversely, the devices in the \textit{Producer} domain also form a
private network. Therefore, the goal is to create an overlay virtual network that unifies
the \textit{Consumer} and the \textit{Producer} domains in way such that the original interests
and contents are only visible to devices inside these two domains. In other words, the interest
$I_e$ and the content $C_e$, that are forwarded outside these domains, should give no information
about the original interest $I_p$ and the actual content $C_p$.

In order to achieve such private communication, $G_c$ is introduced in the \textit{Consumer} domain
to encapsulate outgoing interests. Conversely, $G_p$ is responsible for decapsulating the incoming
encapsulated interests and forwarding them in their original form. When the content is forwarded back
in response to the interest, $G_p$ encrypts the content before sending it outside the \textit{Producer}
domain. Finally, $G_c$ decrypts the received content to its original form and forwards it back towards
the \textit{Consumer}. Through the rest of this section, we provide a detailed description on how
such actions are implemented.

Upon the arrival of a new interest $I_p$, $G_c$ performs a lookup in its FIB to verify if the interest
prefix is in the list of prefixes for VPN communication. We assume that the FIB must be pre-configured
with the list of prefixes which will be tunnelled across the VPN, i.e, prefixes associated with the
producers inside of the Producer domain in Fig.~\ref{fig:ccvpn}. Associated with such prefixes
are $G_p$'s name and public key ($pk_e$). Therefore, if the prefix of $I_p$ is in the VPN prefixes
list, $G_P$ runs Algorithm~\ref{alg:interestEncap} to generate a new interest $I_e$ which encapsulates
the original interest $I_p$. Firstly, Algorithm~\ref{alg:interestEncap} generates a random symmetric
key ($k_r$) which will be used later on to perform the \textit{Content} Encryption/Decryption. Next,
it retrieves $G_p$'s name and public key from the FIB. It uses the public key to encrypt the
symmetric key $k_r$ and the original interest $I_p$. Then it creates the new interest $I_e$ with
$G_p$'s name as the interest name and the generated ciphertext as payload. Since $I_e$ has $G_p$'s name,
it will be routed towards $G_p$ and since the payload is singed with $G_p$'s public key, only $G_p$
can retrieve the original interest $I_p$ and the symmetric key $k_r$.

% \begin{framed}
\begin{algorithm}\label{alg:interestEncap}
\SetKwInOut{Input}{input}
\SetKwInOut{Output}{output}
\Input{Original interest $I_p$;}
\Output{Encapsulated interest $I_e$;}
%$Ip_{name}$ = getName($I_p$)\\
%storeToPIT($Ip_{name}$)\\
$k_r$ = symmKeyGen();\\
$Gp_{name}$ = retrieveNameFromFIB($I_p$)\\
$pk_e$ = retrievePKFromFIB($I_p$)\\
$payload$ = $Enc_{pk_e}(I_p||k_r)$\\
$I_e$ = createNewInterest($Gp_{name}$, $payload$)\\
storeToPIT($I_e$,$k_r$)\\
\Return $I_e$;\\
\caption{Interest encapsulation (runs on $G_c$)}
\end{algorithm}
% \end{framed}

After the message is routed towards $G_p$, $G_p$ verifies if the incoming interest name prefix
matches its own name and, if it does, $G_p$ runs Algorithm~\ref{alg:interestDecap}, using its own
secret key ($sk_e$) to decrypt $I_e$'s payload, which results in the original interest $I_p$ and
the symmetric key $k_r$. $G_p$ then stores $I_e$'s name and the symmetric key $k_r$ in its own PIT,
within the entry for the pending interest $I_p$, and forwards $I_p$. $I_e$'s name and $k_r$ are
stored so that they can be used later on to generate the encrypted content $C_e$.

\begin{algorithm}\label{alg:interestDecap}
\SetKwInOut{Input}{input}
\SetKwInOut{Output}{output}
\Input{ Encapsulated interest $I_e$;}
\Input{ Private key $sk_e$;}
\Output{Original interest $I_p$;}
$Ie_{name}$ = getName($I_e$)\\
$cipherText$ = getPayload($I_e$)\\
$I_p||k_r$ = $Dec_{sk_e}(cipherText)$\\
storeToPIT($I_p$,$k_r$,$Ie_{name}$)\\
\Return $I_p$;\\
\caption{Interest decapsulation (runs on $G_p$)}
\end{algorithm}

The original interest $I_p$ is forwarded inside the \textit{Producer} domain until it reaches
the \textit{Producer}. The \textit{Producer} responds with the content $C_p$ which is forwarded back to $G_p$.
Upon receiving $C_p$, $G_p$ fetches for $C_p$'s name (which is equal to $I_p$'s name) on its
PIT, retrieving $k_r$ and $Ie$'s name. Then it uses $k_r$ to encrypt-then-MAC the real content
response $C_p$ and creates $C_e$ which must have the same name as $I_e$ and the encryption of
$C_p$ as payload (Algorithm~\ref{alg:contentEnc}). Since only $G_c$ and $G_p$ share the
symmetric key $k_r$, only $G_c$ will be able to decrypt $C_e$'s payload into $C_p$. Therefore
nobody from outside the VPN is able to access the content nor the Producer's identity.

\begin{algorithm}\label{alg:contentEnc}
\SetKwInOut{Input}{input}
\SetKwInOut{Output}{output}
\Input{Original content $C_p$;}
\Output{Encrypted content $C_e$;}
$name$ = getName($C_p$)\\
$k_r$ = retrieveKeyFromPIT($name$)\\
$Ie_{name}$ = retrieveNameFromPIT($name$)\\
$payload$ = $EncryptThenMAC_{k_r}(C_p)$\\
$C_e$ = createNewContent($Ie_{name}$,$payload$)\\
\Return $C_e$;\\
\caption{Content encryption  (runs on $G_p$)}
\end{algorithm}

Since $C_e$ and $I_e$ have the same name, $C_e$ will be forwarded all the way back to $G_c$.
When $G_c$ receives $C_e$, $G_c$ will execute Algorithm~\ref{alg:contentDec}. It will match
$C_e$'s name to the pending interest $I_e$ in its PIT, retrieving $k_r$. $k_r$ can then be
used to verify the integrity of the received content and to decrypt it into the actual
content $C_p$. After that, $C_p$ can be forwarded back to the \textit{Consumer}. If the MAC
verification fails, it means that $C_e$ has been inadvertantly modified or tampered,
so $G_c$ discards it.

\begin{algorithm}\label{alg:contentDec}
\SetKwInOut{Input}{input}
\SetKwInOut{Output}{output}
\Input{Encrypted content $C_e$;}
\Output{Original content $C_p$;}
$Ce_{name}$ = getName($C_e$)\\
$k_r$ = retrieveKeyFromPIT($Ce_{name}$)\\
$cipherText$ = getPayload($C_e$)\\
$C_p$ = Dec($k_r$,$cipherText$)\\
\eIf{$C_p$ == $\perp$}
    {
    \tcc{MAC verification failed}
    \Return ;\\
    }
    {\Return $C_p$;\\}
\caption{Content decryption (runs on $G_c$)}
\end{algorithm}

For clarity, we have defined a \textit{Consumer} and a \textit{Producer} domain. However,
in reality, each of the two gateways would run both $G_c$ and $G_p$ processes. Therefore,
consumers and producers can exist in both the domains and interests for contents can be
issued from both sides. Also, it is worth to mention that, inside CCVPNs, content caching
would work just as it works in regular CCNs, i.e., routers would be able to cache content
and respond to interests that were previously requested, enabling better resource usage and
lower communication delays. Finally, we emphasize that $G_c$ encapsulation and decryption
functions can also run on Consumer host. Conversely, $G_p$ can be implemented within
the \textit{Producer} host. This enables private one-to-one, many-to-one, and one-to-many communication.

We also consider a slightly different version of CCVPN, in which we assume that $G_c$ and $G_p$
share a pre-installed symmetric key. The only difference in this version is that
Algorithm \ref{alg:interestEncap} uses symmetric key encryption instead of PKE to encapsulate
the original interest. Through the rest of this text, we refer to this variation as symmetric
key version of CCVPN.
